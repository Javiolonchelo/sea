\documentclass[12pt, a4paper]{article}

\usepackage[spanish]{babel}
\usepackage{amsmath}
\usepackage{mathtools}
\usepackage{multicol}
% \usepackage{esvect}
\usepackage{physics}
\usepackage{parskip}

% Bold vectors
\renewcommand{\vec}[1]{\mathbf{#1}}
\makeatletter
\newcommand{\vv}[2][]{
    \@ifempty{#1}{\vec{#2}}{\vec{#2}_{#1}}
}
\makeatother

\title{Apuntes de Sistemas Electroacústicos}
\author{Javier Rodrigo López}
\date{\today}
 

%%%%%%%%%%%%%%%%%%%%%%%%%%%%%%%%%%%%%%%%%%%%%%%%%%%%%
\begin{document}
% \renewcommand{\arraystretch}{1.2}
\maketitle

% Table of contents
\tableofcontents

\newpage
\section*{Introducción}

\section{Altavoces}

\subsection{Introducción al altavoz electrodinámico}


Un altavoz se compone de varios elementos:


\subsubsection{Sección eléctrica}

\subsubsection{Sección de conversión electromećanica}

\subsubsection{Sección mecánica}

Las partes mécanicas del altavoz son la membrana y la araña (suspensión). En términos de mecánica, este sistema es equivalente a un pistón, un muelle y una masa. VER DIAGRAMA CUADERNO

\begin{equation} \label{eq:ley_hooke}
    F = -kz
\end{equation}

\begin{equation} \label{eq:amortiguamiento}
    F = -d \dv{z}{t}
\end{equation}

Donde $d$ es la disipación de energía por rozamiento.

\begin{equation} \label{eq:masa}
    F = m \dv[2]{z}{t}
\end{equation}

\begin{equation} \label{eq:ecuacion_seccion_mecanica}
    F - k z - d \dv{z}{t} = m \dv[2]{z}{t}
\end{equation}

\begin{equation} \label{eq:ecuacion_seccion_mecanica_2}
    F = kz + d \dv{z}{t} + m \dv[2]{z}{t}
\end{equation}

\subsubsection{Sección de conversión mecánico-acústica}

Para hallar la impedancia de radiación, recordamos que una impedancia es la relación entre tensión y corriente. En nuestro modelo, la tensión se relaciona con la velocidad y la corriente con la fuerza. Por lo tanto, la impedancia de radiación es la relación entre la velocidad de la membrana y la fuerza que se ejerce sobre ella.

La impedancia acústica mide cuánto se está oponiendo el medio acústico (aire) al movimiento del pistón. El pistón comprime y expande el aire y genera una variación de la presión.

\begin{equation} \label{eq:incremento_presion}
    \dd p_1 = j\omega\rho_0 v_z d S_1 \frac{e^{-jkr_1}}{2\pi r_1}
\end{equation}

Una región 1 del pistón genera una fuerza sobre otra región 2 del pistón de la siguiente manera: 
\begin{equation} \label{eq:presion_de_1_sobre_2}
    \dd f_{1,2} = j\omega \rho_0 v_z d S_1 \frac{e^{-jkr_1}}{2\pi r_1} \dd S_2
\end{equation}

\begin{equation} \label{eq:presion_de_2_sobre_1}
    F = \frac{j\omega \rho_0 v_z}{2\pi} \int_{S_1} \int_{S_2} \frac{e^{-jkr_{1,2}}}{r_{1,2}} \dd S_1 \dd S_2    
\end{equation}

La solución de esta ecuación es:

\begin{equation} \label{eq:impedancia_piston}
    Z = \frac{F}{v_z} = \pi a^2 \rho_0 c \left[ \left( 1 - \frac{J_1 (2ka)}{ka}  \right) + j \left( \frac{H_1 (2ka)}{ka} \right)\right]
\end{equation}

Donde $J_1$ y $H_1$ son las funciones de Bessel de primer tipo y orden 1, respectivamente.

\subsubsection{Modelo completo del altavoz}

\begin{align*}
    e'_g &= \frac{e_g}{Bl} & R_{MS} &= \frac{1}{d} & R_{AP-L} &= \frac{1}{2} \cdot \frac{9\pi}{\rho_0ca^2} \\
    R'_E 
\end{align*}
\end{document}