\documentclass[a4paper, 8pt]{extarticle}

\usepackage[greek,spanish,es-tabla,es-nodecimaldot]{babel}
\usepackage[a4paper, lmargin=0.2cm,rmargin=0.2cm,tmargin=1cm,bmargin=1cm, landscape]{geometry}
\usepackage{multicol}
\usepackage{amsmath}
\usepackage{mathtools}
\usepackage[exponent-product = \cdot, per-mode = fraction]{siunitx}
\usepackage{cancel}
\usepackage{siunitx}
\usepackage{physics}
\AtBeginDocument{\RenewCommandCopy{\qty}{\SI}}
\usepackage{lipsum}
\usepackage{esvect}
\renewcommand{\vec}[1]{\vv{{#1}}}
\renewcommand{\grad}{\vec{\nabla}}
\renewcommand{\cot}{\operatorname{cotg}}

\usepackage{parskip}


\usepackage{tikz}
\usetikzlibrary{arrows, babel, fadings}

\usepackage{circuitikz}
\ctikzset{bipoles/length=1cm}

\usepackage{lmodern}
% \renewcommand{\familydefault}{\sfdefault}
% \renewcommand{\rmdefault}{\sfdefault}

\usepackage{titlesec}
\titleformat{\section}
{\color{blue}\normalfont\Large\bfseries}{\thesection}{1em}{}[{\titlerule[0.8pt]}]
\titleformat{\subsection}{\normalfont\large\bfseries}{\thesubsection}{1em}{}

\titlespacing*{\section}{0pt}{12pt plus 4pt minus 2pt}{5pt plus 2pt minus 2pt}
\titlespacing*{\subsection}{0pt}{12pt plus 4pt minus 2pt}{3pt plus 2pt minus 2pt}
\titlespacing*{\subsubsection}{0pt}{12pt plus 4pt minus 2pt}{3pt plus 2pt minus 2pt}

%\allowdisplaybreaks
\setcounter{secnumdepth}{-1}
\setcounter{tocdepth}{-1}

\newcommand{\pag}[1]{\text{(Pág. #1)}}
\DeclareSIUnit\rayl{rayl}
\DeclareSIUnit\dbspl{\text{dB\ensuremath{_{\textup{SPL}}}}}
\DeclareSIUnit\dbsil{\text{dB\ensuremath{_{\textup{SIL}}}}}


%%% INICIO DEL DOCUMENTO %%%
\begin{document}
\setlength{\parskip}{0pt}
\setlength{\parindent}{0pt}
\pagestyle{empty}
\renewcommand{\arraystretch}{1.5}

\begin{multicols}{3}
    \section{Glosario}
    Pista: las variables suelen estar en cursiva, las unidades no.
    \[
        \begin{alignedat}{2}
            S          & \equiv \text{Sensibilidad}  \left[ \unit{\dbspl} \right]                                                  \\
            H          & \equiv \text{Función de transferencia}  \left[ \unit{\dB} \text{ re. }\unit{\volt\per\pascal} \right]     \\
            P          & \equiv \text{Potencia} \left[ \unit{\watt} \right]                                                        \\
            V          & \equiv \text{Tensión en bornes del altavoz (a veces aparece como } E \text{)} \left[ \unit{\volt} \right] \\
            \text{SPL} & \equiv \text{Nivel de presión sonora (a veces aparece como } L \text{)} \left[ \unit{\dbspl} \right]      \\
            Z_n        & \equiv \text{Impedancia nominal del altavoz} \left[ \unit{\ohm} \right]                                   \\
        \end{alignedat}
    \]

    \section{Fórmulas útiles}
    \subsection{Nivel de presión en la banda útil}
    \[
        \text{SPL}_{\Delta \textup{B}} = 10 \log \left( \sum_{i}^{N} 10^{\frac{\text{SPL}_i}{10}} \right)
    \]
    \color{gray}Donde:
    \begin{itemize}
        \item $\Delta \textup{B}$ se refiere a la banda de frecuencias en la que se realiza la suma (generalmente el rango útil del altavoz).
        \item $N$ es el número de bandas.
    \end{itemize}\color{black}

    \subsection{Potencia en la banda útil}
    La potencia se puede sumar directamente:
    \[
        P_{\Delta \textup{B}} = \sum_{i}^{N} P_i
    \]

    \section{Parámetros de altavoces}
    \subsection{Sensibilidad}
    La sensibilidad $S$ de un altavoz se define como el nivel de presión sonora radiada por el altavoz cuando se le excita con una potencia de \qty{1}{\watt}, sobre la impedancia nominal, a \qty{1}{\meter} de distancia y en el eje de máxima radiación.

    \subsubsection{Corrección de la distancia y de la potencia}
    Si la distancia a la que se mide la sensibilidad no es de \qty{1}{\meter} o si la potencia aplicada en la banda útil no es de \qty{1}{\watt}, se puede calcular la sensibilidad como:
    \[
        S \left( \qty{1}{\watt }, \qty{1}{\meter}, \qty{0}{\degree }  \right) = \text{SPL}_{r,P} + 20 \log \left( r \right) - 10 \log \left( P \right)
    \]

    \color{gray}Donde $\text{SPL}_{r,P}$ es el nivel de presión sonora medido a una distancia $r$ tras aplicar una potencia $P$ al altavoz en la banda útil.\color{black}


    \subsection{Potencia disipada en el altavoz}
    Usando la ley de Ohm ($V=I \cdot R$), la potencia queda como:
    \[ P = V \cdot I = \frac{V^2}{Z_n}\]
    \color{gray}Donde:
    \begin{itemize}
        \item $I$ es la corriente que circula por el altavoz (apenas se usará para los primeros temas de SEA).
    \end{itemize}\color{black}

    \vfill\null
    \columnbreak
    \vfill\null
    \columnbreak
    \vfill\null
    \columnbreak
\end{multicols}
\end{document}
